\glsdisablehyper % disable hyperlinks

\newglossarystyle{persian-to-english}{%
%	\glossarystyle{listdotted}% the base style
	% put the glossary in a two column page and description (as in listdotted style) environment:
	\renewenvironment{theglossary}%
		{\begin{multicols}{2}\begingroup \flushleft }%
		{\endgroup \end{multicols}}%
%	\renewenvironment{theglossary}{}{}%
	% have nothing after \begin{theglossary}:
	\renewcommand*{\glossaryheader}{}%
	% have nothing between glossary groups:
	\renewcommand*{\glsgroupheading}[1]{}%
	\renewcommand*{\glsgroupskip}{}%
	% set how each entry should appear: \glossaryentryfield{label}{formatted name}{description}{symbol}{number list}
	\renewcommand*{\glossaryentryfield}[5]{%
		\glstarget{##1}{##2}% persian term
		\dotfill%dots
		\space \lr{##3} \\%
%		\dotfill%
%		\space {##5} \\%translation term
	}%
	% set how sub-entries appear:
	\renewcommand*{\glossarysubentryfield}[6]{%
		\glossaryentryfield{##2}{##3}{##4}{##5}{##6}%
	}%
}
% ========= Glossary styles (put in files) =========
\newglossarystyle{english-to-persian}{%
%	\glossarystyle{listdotted}% the base style
	% put the glossary in a two column page and description (as in listdotted style) environment:
	\renewenvironment{theglossary}%
		{\begin{multicols}{2}\begingroup \flushright }%
		{\endgroup \end{multicols}}%
%	\renewenvironment{theglossary}{\Latin{}}{\Persian{}}%
	% have nothing after \begin{theglossary}:
	\renewcommand*{\glossaryheader}{}%
	% have nothing between glossary groups:
	\renewcommand*{\glsgroupheading}[1]{}%
	\renewcommand*{\glsgroupskip}{}%
	% set how each entry should appear:
	\renewcommand*{\glossaryentryfield}[5]{%
		\glstarget{##1}{##2}% persian term
		\dotfill%dots
		\space \rl{##3} \\%translation term
	}%
	% set how sub-entries appear:
	\renewcommand*{\glossarysubentryfield}[6]{%
		\glossaryentryfield{##2}{##3}{##4}{##5}{##6}%
	}%
}

% ========= GLOSSARIES =========
\newglossary{p2e-terms}{fa.gls}{fa.glo}{واژه‌نامه فارسی به انگلیسی} % persian to english
\newglossary{e2p-terms}{en.gls}{en.glo}{English to Persian Glossary} % english to persian

\newcommand{\newtrans}[3][]{% params: persian, english translations, first optional is a key
\newtranspl[#1]{#2}{#3}{#2‌ها}%
}

\newcommand{\newtranspl}[4][]{% params: persian, english, plural form of persian, first optional is a key
\ifthenelse{\isempty{#1}}{\def\key{#2}}{\def\key{#1}}%
\newglossaryentry{en:\key}{type={e2p-terms}, name={#3}, description={#2}}% english glossary
%	\newglossaryentry{fa:\key}{type={p2e-terms}, name={#2}, description={#3}}% persian glossary
\newglossaryentry{fa:\key}{type={p2e-terms}, name={#2}, plural={#4}, description={#3}}% persian glossary
}
% ========= END OF GLOSSARIES =========

% Show a translation and footnote it.
% Params (the same as \glsdisplayfirst):{text}{description}{symbol}{insert}
% insert can possibly be filled with some notes on the translation.
\newcommand{\showTransFirst}[4]{%  translation for the first time
\ifthenelse{\isempty{#4}}%
{\textit{#1}\LTRfootnote{ #2}}% if #4 is empty (no notes) 
%%%{\textit{#1}\LTRfootnote{{#2} #4}}% if #4 is not empty
{\textiranic{#1}\LTRfootnote{{#2} #4}}% if #4 is not empty
%{\textit{#1}\footnote{ \lr{#2}؛  #4}}% if #4 is not empty
}

\newcommand{\showTrans}[4]{% translation for next times
\ifthenelse{\isempty{#4}}%
{{#1}}% if #4 is empty (no notes) 
{\textit{#1}\footnote{#4}}% if #4 is not empty
}

\defglsdisplayfirst[p2e-terms]{\showTransFirst{#1}{#2}{#3}{#4}}% protect fragile commands
\defglsdisplay[p2e-terms]{\showTrans{#1}{#2}{#3}{#4}}

% Symbol may temporarily used to keep some notes on the translation.
% It must be replaced with a user1 key which now raises error, texlive must be upgraded.
\newcommand{\term}[2][]{%
\glsadd{en:#2}%
\ifthenelse{\isempty{#1}}{\gls{fa:#2}}{\gls{fa:#2}[#1]}%
}
\newcommand{\termpl}[2][]{%
\glsadd{en:#2}%
\ifthenelse{\isempty{#1}}{\glspl{fa:#2}}{\glspl{fa:#2}[#1]}%
}
%=========== Print Glossaries ===============
% see: http://www.parsilatex.com/forum/SMF/index.php?topic=345.0
%\glossarystyle{persian-to-english}
%\def\glossaryname{واژه‌نامه فارسی به انگلیسی}
%\printglossaries

\newcommand{\printpersianglossary}[1][واژه‌نامه فارسی به انگلیسی]{{%
	\phantomsection % hyperref: enable hyperlinking from the table of contents to this point
	\addcontentsline{toc}{chapter}{#1} % add a line in the Table of Contents (first option, toc), it will be like the ones 
	\renewcommand{\glossarymark}[1]{\markboth{##1}}% correct handling of page header
	\printglossary[type={p2e-terms},style={persian-to-english},title={#1}]%
}}


\newcommand{\printenglishglossary}[1][واژه‌نامه انگلیسی به فارسی]{{%
	\phantomsection % hyperref: enable hyperlinking from the table of contents to this point
	\addcontentsline{toc}{chapter}{#1} % add a line in the Table of Contents (first option, toc), it will be like the ones 
	\renewcommand{\glossarymark}[1]{\markboth{##1}}% correct handling of page header
	\begin{latin}%
	\printglossary[type={e2p-terms},style={english-to-persian},title={\rl{#1}}]%
	\end{latin}%
}}

% Reset the first-use flag of the transaltion glossareis
\newcommand{\resettranslations}{\glsresetall[e2p-terms,p2e-terms]}